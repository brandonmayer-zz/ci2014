\documentclass{article}
\usepackage{enumitem}
\usepackage{graphicx}
\usepackage{xcolor}
\usepackage{etoolbox}
\usepackage[normalem]{ulem}
\usepackage{hyperref}
\usepackage{subcaption}
\usepackage{placeins}
\usepackage{url}
%% \usepackage{enumerate}
\author{Brandon A. Mayer}
\title{sci-wms CI-2014 Outline}
\date{}


\newbool{commentson}
\booltrue{commentson}
\ifbool{commentson}
{
  \long\def\commentbam#1{{\color{red}\bf **Brandon: #1**}}
  %%Just in case Brian is compelled to comment.
  \long\def\commentbm#1{{\color{green}\bf **Brian: #1**}}
}
{
  \long\def\commentb#1{{}}
  \long\def\commentm#1{{}}
}

\begin{document}
\maketitle
\section{Possible Titles}
\begin{enumerate}
  \item sci-wms: A python-based web map service for visualizing geospatial data
\end{enumerate}

\section{Outline}
\begin{enumerate}
  \item U.S. IOOS Costal and Ocean Modeling (COMT) Testbed~\cite{luettich13}
    \begin{enumerate}[label*=\arabic*.]\
      \item {\em Vision}: Pg. 2 Paragraph 2
      \item "Increase Accuracy, reliability and scope of the federal
        suite of operational coastal and ocean modeling products to
        meet the needs of a diverse user community. Operational use
        covers a wide range of society-critical applications including
        forecasts, forensic studies, risk assessment, design and system
        management."
      \item An integral component of realizing the COMT vision are
        visualiztion tools that can enable rapid qualitative assesment
        of coastal maritime experiments potentially conducted by
        disparate entities, using different models, platforms and file
        formats. Sci-wms is a critical cyberinfastructure solution facilitating such a tool.
    \end{enumerate}

  \item Web Map Service (WMS)~\cite{wms14}
    \begin{enumerate}[label*=\arabic*.]
      \item A standard developed by the Open Geospatial Consortium for
        delivering rasterized visual content in response to http
        requests.
      \item Standardizing the http interface separate users from
        datasets, allowing custom queries to be generated by users to
        which a Web Map Service can respond to by generating custom
        visual content.
    \end{enumerate}

  \item sci-wms as a flexible Web Map Service (WMS)
    \begin{enumerate}[label*=\arabic*.]
      \item Sci-wms is a python based implementation of the OpenGIS Web Map Service~\cite{wms14}
      \item sci-wms implements the OpenGIS WMS standard~\cite{wms14}
        using the Django web application framework to handle http
        requests and responses and NumPy and
        Matplotlib~\cite{numpy11,hunter07} for generating visual content.
      \item Sci-wms is implemented using the django web framework with
        numpy~\cite{numpy11} and matplotlib~\cite{hunter07} as a backend for data manipulation and
        visualization.
    \end{enumerate}

    \item Separating Topology and Variable Data
      \begin{enumerate}[label*=\arabic*.]
        \item Costal forcasting and simulation models can contain many
          hundreds of Gigabytes of data, as such, an important design
          challenge for sciwms is to replicate as little data as
          possible.
        \item A conflicting operational requirement is that when
          content is requested by a client, sci-wms must respond in
          real-time to satisfy the request for visual content.
        \item This is achieved by viewing a dataset as consisting of
          two separate entities, a topology and set of model variables.
        \item 
        \item Therefore rather than reproducing model data locally,
          sciwms maintains a database of OpenDAP endpoints for
          accessing model data remotely.
       
        \item Costal modeling experiments are conducted on a
          geo-registered grid. The type of which defines the model
          topology.
        \item Example topologies are curvilinear, rectilinear, regular, and unstructured.
        \item Define a view in the sci-wms context as a rendering of data at a particular geo-location.
        \item When users request a view from sci-wms via the WMS http protocol sciwms
        \item Sci-wms takes the unique approach of separating topology from variable data.

        \item Costal modeling experiments are conducted on a
          geo-registered grid. The type of which defines the model
          topology.
        \item Example topologies are curvilinear, rectilinear, regular, and unstructured.
        \item Define a view in the sci-wms context as a rendering of data at a particular geo-location.
        \item When users request a view from sci-wms via the WMS http protocol sciwms
      \end{enumerate}
    \item NetCDF Markup Language (NCML)
      \begin{enumerate}[label*=\arabic*.]
        \item Virtual Data layer
        \item Provides CF-Compliant facad for raw model output.
        \item Exposes each dataset, which may consist of numerous
          output files in a diverse format as a single CF-Compliant
          dataset accessable via OpenDAP as a netCDF data structure.
      \end{enumerate}

    \item NetCDF CF-Metadata Conventions
      \begin{enumerate}[label*=\arabic*.]
        \item Climate and Forcast (CF) - Metadata Convnetions
        \item Designed to share files using the netCDF API
      \end{enumerate}
\end{enumerate}

\nocite{pyugrid}
\nocite{wms14}
\bibliographystyle{ieeetr}
\bibliography{ci_mayer}

\end{document}




