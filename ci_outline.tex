\documentclass{article}
\usepackage{enumitem}
\usepackage{graphicx}
\usepackage{xcolor}
\usepackage{etoolbox}
\usepackage[normalem]{ulem}
\usepackage{hyperref}
\usepackage{subcaption}
\usepackage{placeins}
\usepackage{url}
%% \usepackage{enumerate}
\author{Brandon A. Mayer}
\title{sci-wms CI-2014 Outline}
\date{}


\newbool{commentson}
\booltrue{commentson}
\ifbool{commentson}
{
  \long\def\commentbam#1{{\color{red}\bf **Brandon: #1**}}
  %%Just in case Brian is compelled to comment.
  \long\def\commentbm#1{{\color{green}\bf **Brian: #1**}}
}
{
  \long\def\commentb#1{{}}
  \long\def\commentm#1{{}}
}

\begin{document}
\maketitle
\section{Possible Titles}
\begin{enumerate}
  \item sci-wms: A python-based web map service for visualizing geospatial data
\end{enumerate}

\section{Outline}
\begin{enumerate}
  \item U.S. IOOS Costal and Ocean Modeling (COMT) Testbed~\cite{luettich13}
    \begin{enumerate}[label*=\arabic*.]\
      \item {\em Vision}: Pg. 2 Paragraph 2
      \item "Increase Accuracy, reliability and scope of the federal
        suite of operational coastal and ocean modeling products to
        meet the needs of a diverse user community. Operational use
        covers a wide range of society-critical applications including
        forecasts, forensic studies, risk assessment, design and system
        management."
      \item An integral component of realizing the COMT vision are
        visualiztion tools that can enable rapid qualitative assesment
        of coastal maritime experiments potentially conducted by
        disparate entities, using different models, platforms and file
        formats. Sci-wms is a critical cyberinfastructure solution facilitating such a tool.
    \end{enumerate}
  \item Web Map Service (WMS)~\cite{wms14}
    \begin{enumerate}[label*=\arabic*.]
      \item A standard developed by the Open Geospatial Consortium for
        delivering rasterized visual content in response to http
        requests.
      \item Standardizing the http interface separate users from
        datasets, allowing custom queries to be generated by users to
        which a Web Map Service can respond to by generating custom
        visual content.
    \end{enumerate}
  \item sci-wms as a flexible Web Map Service (WMS)
    \begin{enumerate}[label*=\arabic*.]
      \item sci-wms implements the OpenGIS WMS standard~\cite{wms14}
        using the Django web application framework to handle http
        requests and responses and NumPy and
        Matplotlib~\cite{numpy11,hunter07} for generating visual content.
      \item Sci-wms is a python based implementation of the OpenGIS Web Map Service~\cite{wms14}
      \item Sci-wms is implemented using the django web framework with
        numpy~\cite{numpy11} and matplotlib~\cite{hunter07} as a backend for data manipulation and
        visualization.
    \end{enumerate}
    \item Separating Topology and Variable Data
      \begin{enumerate}[label*=\arabic*.]
        \item blah
      \end{enumerate}
  %% \item sci-wms as an integral tool for realizing the U.S. IOOS COMT Model Testbed Vision.~\cite{luettich13, luettich12}
\end{enumerate}

\nocite{pyugrid}
\nocite{wms14}
\bibliographystyle{ieeetr}
\bibliography{ci_mayer}

\end{document}




