\subsection{Data Model}
Visualization tools often decompose data into structure and
attributes~\cite{vtk}. Any continuous function to be represented by a
digital computer must be measured at a discrete set of samples while
rendering numerical data typically requires knowledge of the values
between samples to produce perceptually continuous images from
arbitrary viewpoints. Structure encapsulates both the locations and
connectivity relations onto which attributes are superimposed, where
connectivity serves to constrain the interpolation problem. Note that
some authors continue the abstraction of structure into topology and
geometry~\cite{weiler}, however in the context of this research,
topology is synonymous with structure. Figure~\ref{fig:data_hierarchy}
outlines the data model adopted by \sciwms{}. A dataset is composed of
attributes with associated structure which is further classified as a
regular or irregular, known as \cgrid{} and \ugrid{} topologies in
\sciwms{} documentation.
\begin{figure}[ht!]
  \centering
  \begin{subfigure}[t]{0.45\textwidth}
    \centering
    \includegraphics[height=1.5in]{../figs/data_model_hierarchy}
    \caption{}
    \label{fig:data_hierarchy}
  \end{subfigure}
  \begin{subfigure}[t]{0.45\textwidth}
    \centering
    \includegraphics[height=1.5in]{../figs/sciwms_book_db_topology_endpoint_chart}
    \caption{}
    \label{fig:sciwms_topology_endpoints}
  \end{subfigure}
  \caption{(a) Decomposition hierarchy of the data model.  (b)
    \Sciwms{} topology and endpoint data store.}
\end{figure}
