\subsection{Dataset Abstractions}
Visualization tools often decompose data into structure and
attributes~\cite{vtk}. Any continuous function to be represented by a
digital computer must be measured at a discrete set of
samples. However, rendering scientific data typically requires
knowledge of the values between samples to produce a perceptually
continuous images from arbitrary viewpoints. Structure encapsulates
both the locations and connectivity relations onto which attributes
are superimposed, where connectivity serves to constrain the
interpolation problem. Note that some authors further decompose
structure further into topology and geometry~\cite{weiler}, however,
in the context of this research, topology is synonymous with
structure. Figure~\ref{fig:data_hierarchy} outlines the
data model adopted by \sciwms{}. A dataset is composed of attributes
with associated structure that is further classified as a regular or
irregular, known as \cgrid{} and \ugrid{} topologies in
\sciwms{} documentation.

\begin{figure}[ht!]
  \centering
  \begin{subfigure}[t]{0.45\textwidth}
    \includegraphics[width=\textwidth]{../figs/data_model_hierarchy}
    \caption{}
    \label{fig:data_hierarchy}
  \end{subfigure}
  \begin{subfigure}[t]{0.45\textwidth}
    \includegraphics[height=2in]{../figs/sciwms_book_db_topology_endpoint_chart}
    \caption{}
    \label{fig:sciwms_topology_endpoints}
  \end{subfigure}
  \caption{(a) Decomposition hierarchy of the data model. A dataset
    submitted to \sciwms{} is decomposed into attributes and structure
    which is further classified as a regular (\cgrid{}) or irregular
    (\ugrid{}) topologies. (b) \Sciwms{} topology and endpoint data
    store. Topolgies are stored locally in implicit form for \cgrid{}
    or binary R-Tree databases for \ugrid{} topologies. }
\end{figure}
