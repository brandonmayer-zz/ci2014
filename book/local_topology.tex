%%this is ci2014/book/local_topology.tex
\subsection{Local Topology Cache}
Visualization tools often decompose data into structure and
attributes.  Structure encapsulates both the locations (where) and
geometry (connectivity) relations between attributes and constrain the
visualization task. Similarly, \sciwms{} decomposes an externally
hosted dataset into a structure (topology), defined as a
geo-referenced spatial set of locations and geometries (connectivity
of locations), and the underlying data layer. 

For example, atmospheric and oceanagrphic models typically define a
fixed topology covering a particular spatial extent of the earth. A
model will extimate attributes of interest such as sea-surface-height,
wind or current magnitudes and directions. The topology of the model
referrs to positions and connectivity of the locations for which
attributes are to be computed. While a model produces data associated
with every location specified by the topology, visualizations are
typically generated for a subset of the topology (a region of
interest) and a single attribute such as current direction. It is
therefore paramount to the efficiency of visualization software to
represent topologies in such a way as to optimize topology storage and
reduction to facilitate efficient attribute retrieval. 

Topologies can be further classified into one of two categories,
structured and unstructured, referred to as {\bf c-grid} and {\bf
  u-grid} topologies respectively in \sciwms{}
nomenclature. Structured topologies can be represented using
analytically while unstructured topologies require explicit
enumeration. These differences admit different optimal data structures
and algorithms for storage and processing.

To this end, when an dataset endpoint is submitted to \sciwms{}, the
topology of the underlying endpoint is stored locally to \sciwms{} and
a database of topology-endpoint associations are maintained as
visualized in Figure~\ref{fig:sciwms_topology_endpoints}. 

\subsection{Structured (c-grid) Topologies}
Structured or {\bf c-grid} topologies refer geo-referenced locations
and geometries that can be analytically specified, e.g., rectilinear
or curvilinear grids. Storing \cgrid{} topologies amount to storing
the closed form formula. When an analyst requests a view of a
particular region, the subseting
