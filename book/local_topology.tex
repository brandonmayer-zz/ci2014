%%this is ci2014/book/local_topology.tex
\subsection{Local Topology Cache}

\sciwms{} adopts the \cfugrid{} conventions for implementing the data
model. A topology is always embedded in either $\mathbb{R}^1$,
$\mathbb{R}^2$ or, $\mathbb{R}^3$ where the dimension of a topology
summarizes the connectivity of coordinate locations within the ambient
space. A topology with dimension 0 is a set of disconnected points
called \textbf{\textit{nodes}}, a 1D topology consists of lines or curved
boundaries known as \textbf{\textit{edges}} in \cfugrid{} documentation, a 2D
topology is a set of planes or surfaces enclosed by a set of edges
(e.g. triangulation) and generally called \textbf{\textit{faces}}, and
a 3D topology specifies a volume enclosed by a set of faces and are
referred to as \textbf{\textit{volumes}}.

Attributes are defined as numerical quantities that are associated
with the topology. For example, an atmospheric model may estimate wind
direction at the vertices of a triangulated 2D topology or may specify
air temperature at the centroid of cell volumes specified by a 3D
topology. Additionally, attributes maintain their own
dimensionallity. An attribute specifying temperature or
sea-surface-height for example would consist of an array of scalars
while wind directions are vector valued and tensor valued attributes
are also possible.
