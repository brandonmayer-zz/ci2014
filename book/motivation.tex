%% this is /book/motivation.tex
\section{Motivation}
\label{sec:motivation}
Due to the explosion of atmospheric, oceanagraphic, climate and
weather data, it is no longer feasible for a single institution to
host and maintain a centralized datastore. Modern data managment has
been shifting hosting and maintenance responsibilities of a large
datasets to multiple participating institutions unified by a indexing
service exposing the participating agencies associated data and
metadata. A common practice is for data producing organizations to
host their own data which is then made available to a central agency
via a suit of web services. The central agency then provides a unified
view of the aggregated dataset to end users by compiling a catalogue
of the external data. Such federated datasets may span petabytes of
geospatial information, be composed of millions of files in different
formats all generated and hosted by vastly different systems located
across the globe. Additionally, the catalogue can grow or shrink as
new datasets or participants are added and removed from the
federation. By interacting with the catalogue, the end user (such as
an analyst) can search through the aggregated meta data and download
particular datasets of interest, agnostic to the exact locations of
the endpoints abstracted by the central service.

While a decentralized approach to data management offers many
advantages such as robustness to failure (a failure at any one
organization only effects the data associated with that organization),
data reduction and analysis tools have been slower to adapt to the new
framework. For example, there exist many programs to visualize
cartographic data on a single machine and many such programs are
designed to deal with output of specific applications or data saved in
a particular file format. This type of workflow is ripe for different
analysts to download local copies of the same datasets to different
machines and use different tools to create incompatible visualizations
or comparisons of the same data.

\sciwms{} is a web service designed for visualizing federated
data. Specifically, \sciwms{} maintains a list of http endpoints and a local cache of information for producing visualizations in realtime.


The U.S. Integrated Ocean Observing System (\ioos{}) Coastal and Ocean
Modeling Testbed (\comt{}) was formed to unify otherwise disparate
entities in government, academia and industry to leverage the
proliferation of oceanagraphic data and modeling techniques to combat
natural and man-made coastal stressors by accelerating the turnaround
from research and development to operational application of
society-critical applications including: forecasting, model
comparison, model skill assessment, and algorithmic/parameterization
improvements~\cite{luettich13}. Key to the U.S. \ioos{} \comt{}
mission is an extensible and universally available tool for quickly
visualizing and assessing a diverse set of coastal modeling
data. \sciwms{} is a general \ogc{} \wms{} solution for serving
rasterized visualizations of geospatial data which has been deployed
for the \comt{} project to provide visualizations of a wide range of
scientific data.
