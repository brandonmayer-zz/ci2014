%% this is /book/motivation.tex
\section{Motivation}
\label{sec:motivation}
Due to the explosion in the amount of atmospheric, oceanographic,
climate and weather data either recorded \emph{in-situ} or generated
by modeling, inference and prediction algorithms, it is no longer
feasible for a single institution to host and maintain a centralized
database of information. Modern data management has been shifting
hosting and maintenance responsibilities of large datasets to multiple
participating institutions unified by a catalog service which provides
a single view of the distributed data to end users~\cite{luettich13,
  williams09, chervenak00}. The institution responsible for
maintaining the catalog compiles \metadata{} regarding data hosted and
exposed to the catalog by registered data producers. Such federated
datasets potentially span petabytes of information, may be
composed of millions of files in different formats and are typically
generated and hosted by vastly different systems located across the
globe. End users (such as analysts or scientists) interface with the
catalog to search through the aggregated \metadata{} and interact with
particular data of interest, agnostic to distributed nature of the
database.

Although a decentralized approach to data management offers many
advantages, data reduction and analysis tools have been slow to adapt
to the distributed framework. There exists an abundance of
applications for visualizing cartographic data on local computing
resources, requiring analysts to download local copies of datasets
and potentially reformat the data into the appropriate file format
before processing and analysis can begin. Even if an end user has
access to sufficient resources to download and process a dataset of
interest, tools designed for centralized local systems increase
project costs in terms of bandwidth usage, time and
storage. Additionally, coupling centralized processing with
decentralized storage introduces the risk that different analysts
working with identical local copies of data obtained from the same
federation may use different local programs to generate incompatible
visualizations and reach conflicting conclusions. Normalizing these
results introduces a potential point of error and likewise increases
project costs in terms of time and accuracy.

%% HOW DOES SCIWMS SOLVE THESE PROBLEMS
%% CONTRIBUTIONS CAN GO HERE
By providing a single web-based visualization tool, \sciwms{} fills a
significant gap in cooperative and distributed computing. Redundancy
is minimized by avoiding dataset replication as minimal data is
fetched as needed to fulfill each visualization request. Additionally,
\sciwms{} enables scientists to visualize data through a simple web
browser, facilitating rapid and consistent algorithmic and parametric
comparisons. Since \sciwms{} may be deployed on a server with
dedicated hardware for storage, processing and visualization,
\sciwms{} lowers costs and barriers to entry for analysts who would
otherwise have to invest in the necessary cyberinfrastructure to
download and visualize local copies of distributed data.

While there exists other distributed geospatial visualization
solutions including \ncwms{}~\cite{blower13},
\mapserver{}~\cite{mapserver14}, and
\geoserver{}~\cite{geoserver14}. \sciwms{} and \ncwms{} are the only
platforms which support \netcdf{}, a community standard file
format. Additionally, \sciwms{} is the only service which supports
data associated with unstructured topologies as outlined in section~\ref{sec:ugrid}.

%% Furthermore, after a catalog service is registered with \sciwms{},
%% data producers need only publish \metadata{} about a new dataset to the
%% catalog and make their data available over http. \sciwms{} will
%% automatically add new catalog entries to the list of datasets it can
%% visualize transparent to the end user.

%% \sciwms{} represents a significant advance in geospatial
%% science by providing a platform for producing consistent
%% visualizations of distributed data. \sciwms{} is a visualization
%% system with a distributed memory model, minimizing data redundancy by
%% fetching the minimal amount of data needed to fulfill a request for a
%% visualization from the data federation.

%% While \sciwms{} implements the \ogc{}-\wms{} compliant, it is
%% augmented with services for interacting with standard meta-data
%% catalogs such as Catalog Service for the Web (\csw{})~\cite{csw14},
%% allowing \sciwms{} to autonomously track dynamic catalogs. Once a
%% catalog service is registered to \sciwms{}, any data a producer
%% publishes to the catalog service is automatically ingested by
%% \sciwms{}. \sciwms{} does not replicate the data hosted by the
%% publisher but instead parses meta data and creates a compressed local
%% cache to facilitate real-time data visualization.

%% \sciwms{} embraces distributed database principles and promotes the
%% separation of concerns software and project management practice. For
%% example, \sciwms{} may be deployed by a member of a much larger
%% project who's sole responsibility is providing a visualization
%% platform, promoting quality through specialization, saving time and
%% costs for data analysis projects by providing a simple interface for
%% end users to generate consistent visualizations of federated data.
