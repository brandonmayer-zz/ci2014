%% this is /book/motivation.tex
\section{Motivation}
\label{sec:motivation}
Due to the explosion in the amount of atmospheric, oceanographic,
climate and weather data either recorded \emph{in-situ} or generated
by modeling, inference and prediction algorithms, it is no longer
feasible for a single institution to host and maintain a centralized
database of information. Modern data management has been shifting
hosting and maintenance responsibilities of large datasets to multiple
participating institutions unified by a catalog service which provides
a single view of the distributed data to end users~\cite{luettich13,
  williams09, chervenak00}. The institution responsible for
maintaining the catalog compiles \metadata{} regarding externally
hosted data, exposed to the catalog by registered data producers. Such
federated datasets potentially span petabytes of information, may be
composed of millions of files in different formats and are typically
generated and hosted by vastly different systems located across the
globe. End users (such as analysts or scientists) interface with the
catalog to search through the aggregated \metadata{} and interact with
particular data of interest, agnostic to distributed nature of the
database.

Although a decentralized approach to data management offers many
advantages, data reduction and analysis tools have been slow to adapt
to the distributed framework. There exists an abundance of
applications for visualizing cartographic data on local computing
resources, requiring analysts to download local copies of datasets
and potentially reformat the data into the appropriate file format
before processing and analysis can begin. Even if an end user has
access to sufficient resources to download and process a dataset of
interest, tools designed for centralized local systems increase
project costs in terms of bandwidth usage, time and
storage. Additionally, coupling centralized processing with
decentralized storage introduces the risk that different analysts
working with identical local copies of data obtained from the same
federation may use different local programs to generate incompatible
visualizations and reach conflicting conclusions. Normalizing these
results introduces a potential point of error and likewise increases
project costs in terms of time and accuracy.

%% HOW DOES SCIWMS SOLVE THESE PROBLEMS
%% CONTRIBUTIONS CAN GO HERE
The Open Geospatial Consotrium (\ogc{}) defines the Web Mapping
Service (\wms{})~\cite{wms14} standard which describes how a compliant
visualization server responds to HTTP requests from a \wms{} compliant
client application. \sciwms{} is an implementation of an \ogcwms{}
server which responds to requests from clients with either \metadata{}
or geo-registered visualizations.  While there exists other \ogcwms{}
compliant solutions including \ncwms{}~\cite{blower13},
\mapserver{}~\cite{mapserver14}, and \geoserver{}~\cite{geoserver14},
\sciwms{} and \ncwms{} are the only platforms which support \netcdf{},
a community standard file format. Additionally, \sciwms{} is the only
\ogcwms{} service to supports data associated with unstructured
topologies as outlined in section~\ref{sec:ugrid}.

\sciwms{} is designed in such a way as to fill significant gaps in
cooperative and distributed geoscientific computing. Data redundancy
is minimized by avoiding dataset replication as minimal data is
fetched as needed to fulfill each visualization request. \sciwms{}
enables the end user generate scientific visualization using \wms{}
clients, which may be simple web browser applications, facilitating
rapid and consistent algorithmic and parametric
comparisons. Furthermore, because \sciwms{} may be deployed on a
server with dedicated hardware for storage, processing and
visualization, \sciwms{} lowers costs and barriers to entry for
analysts who would otherwise have to invest in the necessary
cyberinfrastructure to download and visualize local copies of
distributed data.
