%% this is /book/motivation.tex
\section{Motivation}
\label{sec:motivation}
Due to the explosion of the amount of atmospheric, oceanagraphic,
climate and weather data, either recorded \emph{in-situ} or generated
by modeling, inference or prediction algorithms, it is no longer
feasible for a single institution to host and maintain a centralized
datastore. Modern data management has been shifting hosting and
maintenance responsibilities of large datasets to multiple
participating institutions unified by a catalogue service to provide a
single, unified view of the distributed data to the end user or
analyst.

A common practice is for data producing organizations to host their
own data to be exposed to the catalogue service via a particular
communication protocol. The institution responsible for maintaining
the catalogue then provides a unified view of the aggregated dataset
to end users by compiling meta data associated with the external
datum. Such federated datasets may span petabytes of geospatial
information, be composed of millions of files in different formats all
generated and hosted by vastly different systems located across the
globe. Additionally, the catalogue can grow or shrink as new datasets
or participants are added and removed from the federation. By
interacting with the catalogue, the end user (such as an analyst) can
navigate and search through the aggregated meta data and interact with
particular datasets of interest, agnostic to distributed nature of the
database.

While a decentralized approach to data management offers many
advantages such as robustness to failure (a failure at any one
organization only effects the data associated with that organization),
data reduction and analysis tools have been slower to adapt to the new
framework. For example, there are an abundance of applications for
visualizing cartographic data on a single machine. However, many such
programs are designed to deal with the output of specific modeling
applications or data saved in a particular file format. To use these
tools, analysts must download local copies of datasets to process a
particular dataset with their local tools. This increases the projects
cost in terms of bandwidth usage and time needed to wait for downloads
to finish. Additionally, such a decentralized-local analytical
workflow increases the risk that different analysts working with
identical local copies of data obtained from the same federation may
use different local programs to generate incompatible visualizations
or comparisons of the same data. Normalizing comparisons and
visualizations made with different software introduces another
potential point of error and increases the costs of analysis in terms
of time and accuracy.

 \sciwms{} is a web service designed to solve many of the
aforementioned problems. By maintaining a list of web accessible
endpoints, \sciwms{} is able to transparently produce consistent
visualizations for federated data. While \sciwms{} implements the
\ogc{} \wms{} protocol, it is augmented with services for interacting
with standard meta-data catalogues such as \csw{}~\cite{csw14},
allowing \sciwms{} to autonomously track dynamic federations.

\sciwms{} uses \ncml{} (NetCDF Markup Language) to implement a data
abstraction layer. This allows data hosting agencies to maintain their
own environments and file formats which are exposed to the outside
world without writing custom i/o software or replicating and
reformating each file to a standard format. Additionally, \sciwms{} is
CF-Compliant~\cite{cf}, offering consistent views of endpoints which
adhering to the CF-Metadata conventions embedded in the \ncml{} file.

Local analyis workflow violates the eothos of distributed database
schemes. If the goal of a federated project is to eliminate data
redundency, data processing and visualization software must adapt to
the imposed distributed memory model to likewise minimize data
redundency. \sciwms{} adherse to distributed database principles while
saving costs to data analysis projects by providing a simple interface
for end users to generate consistent visualizations of federated
datasets. Perhaps more importantly, the introduction of a single
web-based visualization service for federated data ensures qualitative
assessments and conclusions are made on equal footing regardless of
analyst or data origin and is one of the first services for
visualizing data associated with an unstructured topology.
