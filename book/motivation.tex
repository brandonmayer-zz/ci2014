%% this is /book/motivation.tex
\section{Motivation}
\label{sec:motivation}
With the explosion of atmospheric, oceanagraphic, climate and weather
data, federated datasets may span petabytes of geospatial information
comprised of millions of files in different generated by multiple
organizations located across the globe, it is no longer feasible to
replicate such externally hosted, but semantically related data in a
centralized repository for processing and visualization.  \sciwms{} is
a visualization tool specifically designed for federated data that
maintains a local database of meta-data about each externally hosted
dataset to avoid data-replication while providing rapid
generalization of visualizations.

The U.S. Integrated Ocean Observing System (\ioos{}) Coastal and Ocean
Modeling Testbed (\comt{}) was formed to unify otherwise disparate
entities in government, academia and industry to leverage the
proliferation of oceanagraphic data and modeling techniques to combat
natural and man-made coastal stressors by accelerating the turnaround
from research and development to operational application of
society-critical applications including: forecasting, model
comparison, model skill assessment, and algorithmic/parameterization
improvements~\cite{luettich13}. Key to the U.S. \ioos{} \comt{}
mission is an extensible and universally available tool for quickly
visualizing and assessing a diverse set of coastal modeling
data. \sciwms{} is a general \ogc{} \wms{} solution for serving
rasterized visualizations of geospatial data which has been deployed
for the \comt{} project to provide visualizations of a wide range of
scientific data.
