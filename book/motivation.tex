%% this is /book/motivation.tex
\section{Motivation}
\label{sec:motivation}
Due to the explosion in the amount of atmospheric, oceanagraphic,
climate and weather data, either recorded \emph{in-situ} or generated
by modeling, inference or prediction algorithms, it is no longer
feasible for a single institution to host and maintain a centralized
datastore. Modern data management has been shifting hosting and
maintenance responsibilities of large datasets to multiple
participating institutions unified by a catalogue service to provide a
single, unified view of the distributed data to end users and analysts.

A common practice is for data producing organizations to host their
own data which is exposed to the catalogue service via a particular
communication protocol. The institution responsible for maintaining
the catalogue then provides a unified view of the aggregated dataset
to end users by compiling meta data associated with the participating
organizations and registered data. Such federated datasets may
span petabytes of geospatial information, be composed of millions of
files in different formats generated and hosted by vastly
different systems located across the globe. Additionally, the
catalogue can grow or shrink as new datasets or participants are added
and removed from the federation. By interacting with the catalogue,
the end user (such as an analyst) can navigate and search through the
aggregated meta data and interact with particular data of
interest, agnostic to distributed nature of the database.

While a decentralized approach to data management offers many
advantages such as robustness to failure (a failure at any one
organization only effects the data associated with that organization),
data reduction and analysis tools have been slower to adapt to the new
framework. For example, there are an abundance of applications for
visualizing cartographic data on a single computer or
clusters. However, many such programs are designed to deal with data
saved in a particular \textit{local} file format, requiring analysts
to download a local copy datasets, potentially reformat the data into
the appropriate container before processing and analysis may
begin. This paradigm implicitly assumes every potential end user has
the resources to copy and process a potentially massive amount of
data, which is not always the case. Even if an analyst has access
sufficient resources, tools operating on local systems increase the
project costs in terms of bandwidth usage, time and
storage. Additionally, coupling decentralized storage and local
processing introduces the risk that different analysts working with
identical local copies of data obtained from the same federation may
use different local programs to generate incompatible visualizations
or conclusions. Normalizing these results introduces a potential point
of error and likewise increases the costs of analysis in terms of time
and accuracy. The local analysis paradigm violates the ethos of
distributed data storage as the goal of a federated project is to
minimize data redundency. Therefore, data processing and visualization
software must adapt to the imposed distributed memory model to
likewise to realize the same benefits.

\sciwms{} is a web service designed to solve many of the
aforementioned problems. By maintaining a list of web accessible
endpoints, \sciwms{} is able to transparently produce consistent
visualizations of federated data. While \sciwms{} implements the
\ogc{} \wms{} protocol, it is augmented with services for interacting
with standard meta-data catalogues such as \csw{}~\cite{csw14},
allowing \sciwms{} to autonomously track dynamic federations.

\sciwms{} uses \ncml{} (\netcdf{} Markup Language) to implement a data
abstraction layer. This allows data hosting agencies to maintain their
own environments and file formats which are exposed to \sciwms{}
without writing custom i/o software or replication and reformatting of
information. Additionally, \sciwms{} is CF-Compliant~\cite{cf},
offering consistent views of endpoints which adhering to the
CF-Metadata conventions embedded in the \ncml{} file.

\sciwms{} embraces distributed database principles and promotes the
separation of concerns software and project management
practice. \sciwms{} may be deployed by a member of a much larger
project who's sole responsibility is providing a visualization
platform, promoting quality through specialization. Additionally,
\sciwms{} saves costs for data analysis projects by providing a simple
interface for end users to generate consistent visualizations of
federated data. Perhaps more importantly, the introduction of a single
web-based visualization service for distributed datasets ensures
qualitative assessments and conclusions are made on equal footing
regardless of analyst or data origin and is one of the first services
for visualizing geospatial data associated with an unstructured
topology.
