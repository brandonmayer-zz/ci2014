%% this is /book/motivation.tex
\section{Motivation}
\label{sec:motivation}
Due to the explosion of atmospheric, oceanagraphic, climate and
weather data, it is no longer feasible for a single institution to
host and maintain a centralized datastore. Modern data managment has
been shifting hosting and maintenance responsibilities of a large
datasets to multiple participating institutions unified by a indexing
service exposing the participating agencies associated data and
metadata. A common practice is for data producing organizations to
host their own data which is then made available to a central agency
via a suit of web services. The central agency then provides a unified
view of the aggregated dataset to end users by compiling a catalogue
of the external data. Such federated datasets may span petabytes of
geospatial information, be composed of millions of files in different
formats all generated and hosted by vastly different systems located
across the globe. Additionally, the catalogue can grow or shrink as
new datasets or participants are added and removed from the
federation. By interacting with the catalogue, the end user (such as
an analyst) can search through the aggregated meta data and download
particular datasets of interest, agnostic to the exact locations of
the endpoints abstracted by the central service.

While a decentralized approach to data management offers many
advantages such as robustness to failure (a failure at any one
organization only effects the data associated with that organization),
data reduction and analysis tools have been slower to adapt to the new
framework. For example, there are an abundance of applications for
visualizing cartographic data on a single machine, however, many such
programs are designed to deal with the output of specific modeling
applications or data saved in a particular file format. Combining
localized tools to a decentralized dataset exposes the risk that
different analysts working with identical local copies of data obtained
from the same federated source may use different tools locally to create
incompatible visualizations or comparisons of the same data.

\sciwms{} is a web service designed for visualizing federated
data. Specifically, \sciwms{} maintains a list of web accessible
endpoints and a local cache of information for producing
visualizations in real-time. Additionally, \sciwms{} is augmented with
services for reading standard data-aggregation catalogues such as
\csw{}~\cite{csw}, allowing \sciwms{} to keep track of dynamic
federations. By providing a simple interface for analysts to produce
visualizations of federated data costs are lowered as the analyst
doesn't spend time customizing a specific tool to a local copy of a
single dataset. More importantly, a single web-based tool for
producing visualizations of federated data, ensures qualitative
assessments and conclusions are made on equal footing.
