%%this is /book/sciwms.tex
\section{\sciwms{}}
\label{sec:sciwms}

\subsection{Overview}
\label{sec:overview}
\Sciwms{} is an open-source implementation of the Open Geospatial
Consortium's (\ogc{}) Web Map Service (\wms{}) standard which
specifies an HTTP interface for generating rasterized visualizations
of geospatial data~\cite{wms14}. More specifically, \wms{} defines a
RESTful API which responds to standardized HTTP GET requests from a
web client. A typical \wms{} request specifies a data layer and region
of interest with optional meta data such as rendering style
parameters. A \wms{} response may include additional meta data
providing information about data registered with the service or a
visualization (commonly a PNG or other standard image format) of a
particular data layer for display by the front end web client.

\sciwms{} is implemented in Python using the Django~\cite{django} web
framework and standard cross-platform numerical software,
NumPy~\cite{numpy11} and Matplotlib~\cite{hunter07} for generating
visual content. Additionally, the open-source python implementation
provides a cross-platform \wms{} solution which can leverage the suite
of tools developed by the geospatial data analysis community, such as
pyugrid~\cite{pyugrid}, to maintain pace with the latest geospatial
software and standards developments including unstructured grid
support and \cfugrid{} Compliance~\cite{cfugrid}.

While the \ogc{} \wms{} specification standardizes the client-server
communication protocol, \wms{} implementations free to design that
fulfills \wms{} requests and responses. Vital to the efficiency of
\sciwms{} is the abstraction of a registered dataset into structure
(also known as topology) and attributes as shown in
figure~\ref{fig:data_hierarchy} and detailed in subsequent
sections. \Sciwms{} maintains a local topology cache for efficient
storage and processing of spatial neighborhoods and subsets with
respect to data structure. For storage efficiency, attributes are not
replicated locally but referenced via OGC compliant web-services and a
database of structure-endpoint pairs is maintained as visualized in
figure~\ref{fig:sciwms_topology_endpoints}. Because geospatial \wms{}
requests are commonly restricted to a subset of the Earth's surface,
\sciwms{} uses the topology cache to compute the subset of numerical
attributes needed to fulfill each request prior to retrieving the
appropriate attributes (which are typically external to the server
hosting \sciwms{} and accessed via HTTP). Furthermore, by classifying
a topology as regular or irregular, efficient algorithms and data
structures are exploited to optimize the computation of attribute subsets.
