\section{\sciwms{}}
\label{sec:sciwms}
\Sciwms{}\footnote{\url{https://github.com/brandonmayer/sci-wms}} is
an open-source \python{} implementation of the Open Geospatial
Consortium (\ogc{}) Web Mapping Service (\wms{}) protocol~\cite{wms14}
using standard cross-platform numerical software,
NumPy~\cite{numpy11}, Matplotlib~\cite{hunter07} and
Django~\cite{django} web framework for generating and serving visual
content. The \ogcwms{} specification defines a Representational State
Transfer (REST) API~\cite{Fielding02} which responds to standardized
HTTP GET requests from a \wms{} client for serving rasterized
visualizations of geospatial data. A typical \wms{} request specifies
a data layer and region of interest with optional \metadata such as
rendering style parameters. A \wms{} response may include additional
information regarding the selected data or a visualization in the form
of a PNG or other standard image format. A base \wms{} client has been
developed in \javascript{} which gives analysts the ability to
generate and interact with visualizations using only a web
browser~\cite{comtui}.
%% The \wms{} client is likewise open-source, was used to create the
%% visualizations in the paper, and is currently in active use for
%% U.S. \ioos{} \comt{} project where \sciwms{} is responsible for
%% visualizing over 50 terabytes of distributed geospatial data (see
%% section~\ref{sec:ioos}).
While \sciwms{} is \ogcwms{} compliant, it is augmented with
services for automatically polling and interacting with \ogc{} -
Catalog Service for the Web (\csw{}) compliant met-data
catalogs~\cite{csw14}, allowing \sciwms{} to autonomously track
dynamic distributed datasets.

Data to be visualized by \sciwms{} should be exposed by producers in
one of the many community standard formats for geoscientific gridded
data such as \netcdf{}, \hdf{}/\hdf5, or \grib{}/\grib2 with
accompanying metadata adhering to the CF (Climate and Forecast)
metadata conventions~\cite{cf}.

While the \ogcwms{} specification standardizes client-server
communication, implementations vary dramatically in how a
particular system fulfills the a \wms{} request. Vital to
the efficiency of \sciwms{} is the decomposition of a registered
dataset into \textbf{structure} (also known as \textbf{topology}) and
\textbf{attributes} as shown in figure~\ref{fig:data_hierarchy} and
detailed in subsequent sections. \Sciwms{} maintains a local topology
cache for efficient storage and processing of spatial neighborhoods
and subsets with respect to data structure. To minimize redundancy,
attributes are not replicated locally but referenced via standard
web-services and a database of structure-endpoint pairs is maintained
as visualized in figure~\ref{fig:sciwms_topology_endpoints}. As
geospatial \wms{} requests are commonly restricted to a subset of the
Earth's surface, \sciwms{} uses the topology cache to compute the
subset of numerical attributes needed to fulfill each request prior to
retrieving the appropriate data, typically via HTTP request. Furthermore, by
classifying a topology as regular or irregular, efficient algorithms
and data structures are exploited to optimize the computation of
attribute subsets.
