\section{\sciwms{}}
\label{sec:sciwms}
\Sciwms{} is an open-source implementation of the Open Geospatial
Consortium's (\ogc{}) Web Map Service (\wms{})~\cite{wms14} written in
Python using the Django~\cite{django} web framework and standard
cross-platform numerical software, NumPy~\cite{numpy11} and
Matplotlib~\cite{hunter07} for generating visual content. The \wms{}
specification defines a REST API~\cite{Fielding02} which responds to
standardized HTTP GET requests from a web client for serving
rasterized visualizations of geospatial data. A typical \wms{} request
specifies a data layer and region of interest with optional meta data
such as rendering style parameters. A \wms{} response may include
additional meta data providing information about data registered with
the service or a visualization in the form of a PNG or other standard
image format.

While the \ogc{} \wms{} specification standardizes the client-server
communication protocol, \wms{} implementations vary dramatically in
how a particular system fulfills the \wms{} requests and
responses. Vital to the efficiency of \sciwms{} is the decomposition
of a registered dataset into \textbf{structure} (also known as
\textbf{topology}) and \textbf{attributes} as shown in
figure~\ref{fig:data_hierarchy} and detailed in subsequent
sections. \Sciwms{} maintains a local topology cache for efficient
storage and processing of spatial neighborhoods and subsets with
respect to data structure. To minimize redundancy, attributes are not
replicated locally but referenced via standard web-services and a
database of structure-endpoint pairs is maintained as visualized in
figure~\ref{fig:sciwms_topology_endpoints}. As geospatial \wms{}
requests are commonly restricted to a subset of the Earth's surface,
\sciwms{} uses the topology cache to compute the subset of numerical
attributes needed to fulfill each request prior to retrieving the
appropriate data, which are typically external to the server hosting
\sciwms{} and accessed via HTTP. Furthermore, by classifying a
topology as regular or irregular, efficient algorithms and data
structures are exploited to optimize the computation of attribute
subsets.
