\section{Results}
\label{sec:results}
Figure~\ref{fig:adcirc_comp} shows a web portal utilizing the
\sciwms{} backend to compare ADCIRC model output for Hurricane Ike
with water levels observed by NOAA stations. Ongoing development is in
progress for \sciwms{} to support emerging geophysical datasets such
as ensemble model output and to provide clear visual support for
the assessment and quantification of model skill and performance
metrics.

\begin{figure}[ht!]
  \centering
  \includegraphics[width=\columnwidth]{../figs/SciWMS_ModelObsComparison}
  \caption{Comparison of ADCIRC (unstructured topology) model results
    with observed water levels in the Northern Gulf of Mexico for
    Hurricane Ike. Verified observed water levels are from NOAA's
    Station 8760922 (red dot on map). The map shows modeled water
    levels (in meters above the geoid) at the peak of the storm in
    southern Louisiana. The time series plot shows both the modeled
    (green) and observed (orange) water levels. The vertical blue line
    in the time series plot corresponds to the current time of the
    map.}
  \label{fig:adcirc_comp}
\end{figure}

\begin{figure}[ht!]
  \centering
  \includegraphics[width=\columnwidth]{../figs/vims_selfe_ubaratropic_vbaratropic_chesapeake_bay}
    \caption{SELFE model of current
    direction and speed in the Chesapeake Bay area.}
\end{figure}

\begin{figure}[ht!]
  \centering
  \includegraphics[width=\columnwidth]{../figs/inundation_tropical_VIMS_SELFE_hurricane_rita_2d_final_run_with_waves_sea_surface_wave_significant_height}
  \caption{Visualizing SELFE model of significant sea surface wave height along the eastern coast of the United States. The underlying topology is an unstructured grid with over 5 million nodes which SCI-WMS can handle in real time.}
\end{figure}
