\subsection{\ugrid{} Topolgy}
\label{sec:ugrid}
A \ugrid{} (also called {\bf irregular} or {\bf unstructured})
topology is defined as a set of sample locations with connectivity
relations that do not admit a closed-form representation. Unstructured
topologies offer the highest level of flexibility from a visualization
and modeling standpoint as higher sampling frequencies may be used in
regions of interest while sparsely sampling low-impact areas to
conserve computational resources, but have larger storage and
processing requirements compared to regular topologies. As storage and
processing hardware has become more accessible, unstructured data has
become more prevalent due to the flexibility of the
representation. However, most existing visualization technologies in
the \cf{} community can only render regularly structured
datasets. \sciwms{} is one of the first visualization services to
support rendering irregularly structured data.

Any \ncml{} file exposing the topology of an externally hosted dataset
according to the \cfugrid{} sepcification can be processed by
\sciwms{}. According the \cfugrid{} standard, a topology is always
embedded on the real line, in the plane or space with sample
locations, the vertices of the topology, exposed as an array of
coordinates in the appropriate ambient space. Vertex connectivity is
expressed as an array where each element is an index into the vertex
list. The dimension of a topology defines the atomic spatial element
created by the connectivity list. The \cfugrid{} specification defines
topology dimension recursively: a topology with dimension 0 defines a
set of disconnected points (no connectivity) called
\textbf{\textit{nodes}}, a 1D topology consists of lines or curved
boundaries known as \textbf{\textit{edges}}, a 2D topology is a set of
planes or surfaces enclosed by a set of edges (e.g. triangulation)
called \textbf{\textit{faces}}, and a 3D topology specifies the volume
enclosed by a set of faces called \textbf{\textit{volumes}}.

In contrast with \cgrid{} topologies, \ugrid{} topologies require
explicit enumeration of sample locations and connectivity, requiring
spatially-aware data structures for optimal storage and processing for
performant visualization algorithms. To this end, \sciwms{} maintains
a local topology cache, storing \ugrid{} topologies using binary
\rtree{}~\cite{Guttman84} data structures on disk locally on the
deployment server for fast access. The \rtree{} is created when the
dataset is first registered with the \sciwms{} service and if a change
in the underlying data is detected at an endpoint associated with a
topology cache, the \rtree{} is rebuilt.
